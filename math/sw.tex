\documentclass[letterpaper,11pt]{article}

\usepackage[utf8]{inputenc}
\usepackage[T1]{fontenc}
\usepackage{amsmath}
\usepackage{amsfonts}
\usepackage{amssymb}
\usepackage{algorithm}
\usepackage{algorithmic}
\usepackage{multicol}
\usepackage{xcolor}
\usepackage[margin=0.85in]{geometry}
\usepackage[english]{babel}

\usepackage[hidelinks]{hyperref}

\author{Armando Faz-Hern\'andez}
\title{Explicit Formulas of Shallue \& van de Woestijne Encoding}
\date{Cloudflare Inc.}

\begin{document}
\maketitle
\begin{abstract}
 This document shows explicit formulas for the construction proposed by Andrew Shallue and Christiaan van de Woestijne (SW)~\cite{SW}. We follow the Fouque-Tibouchi~\cite{ft2012} approach for deriving formulas for SW map without loss of generality.
\end{abstract}

\section{Definitions}
Assume $\mathbb{F}$ is a finite field of characteristic larger than 5.

Let $E_{A,B}$ be an elliptic curve in short Weierstrass form:
\begin{equation}
  E/\mathbb{F} \colon y^2 = f(x)= x^3+Ax+B 
\end{equation}
where $4A^3+27B^2\neq 0$.

Let $V_{A,B}$ be an algebraic threefold defined as:
\begin{equation}
 V/\mathbb{F} \colon {x_4}^2 =  f(x_1) f(x_2) f(x_3)
\end{equation}

Let $S_{A,B}$ be a surface defined as:
\begin{equation}
 S/\mathbb{F} \colon \lambda^2 h(u,v)= -f(u)
\end{equation}
where $h(u,v) = u^2+uv+v^2+A$.


Let $C_{a,b,c}$ be a non-degenerate curve defined as:
\begin{equation}
 C/\mathbb{F} \colon az^2+bw^2=c
\end{equation}
such that $a,b,c\neq 0$. Given the point $(z_0=\sqrt{\frac{c}{a}},0)\in C$, such that $\frac{c}{a}$ is a QR, the parametrization of the points on $C$ on variable $t$ is given as:
\begin{equation}
(z(t),w(t)) = \left( z_0+tw(t),-\dfrac{2az_0t}{at^2+b} \right) 
\end{equation}
Proof on Appendix~\ref{app:conic}.

\subsection{Mappings}

SW proved there exists a rational map that given a point in $S$ obtains a point in $V$.
\begin{equation}
 \begin{aligned}
  \psi \colon S &\rightarrow V\\
   (u,v,\lambda) &\mapsto (x_1,x_2,x_3,x_4)
                 = \left(v , -u-v, u+\lambda^2, \frac{f(u+\lambda^ 2)h(u,v)}{\lambda}\right)
 \end{aligned}
\end{equation}
This map is proved for any point $(u,v,\lambda)\in S$ such that $f(u)\neq 0$, which implies that $\lambda\neq 0$ and $h(u,v)\neq 0$.

SW also showed how to construct a point in $S$ by transforming $S$ into a conic $C$ for which, a parametrization on variable $t$ is easy to find.
Then, 
\begin{equation}
 \begin{aligned}
  \phi \colon \mathbb{F} &\rightarrow S\\
   t &\mapsto (u,v(t),\lambda(t))
 \end{aligned}
\end{equation}
for some fixed $u$ such that $f(u)\neq 0$.


\section{Mathematical Construction}

The mapping $\mathbb{F}\overbrace{\rightarrow C\rightarrow}^\phi S \xrightarrow{\psi} V$ gives a point in $t\mapsto\psi(\phi(t))=(x_1,x_2,x_3,x_4)\in V$. Then, it is guaranteed that exists $i\in\{1,2,3\}$ such that $(x_i,y=\sqrt{f(x_i)})\in E$ is a point on the elliptic curve.
\begin{itemize}
 \item[Note 1.] The composition of each rational map require some conditions to hold.
\item[Note 2.] There are different ways to convert $S$ into $C_{a,b,c}$.
\item[Note 3.] The sign of $y$ must be explicitly chosen.
\end{itemize}

\section{Explicit Formulas}

The purpose of this section is to obtain explicit formulas for $t\mapsto\psi(\phi(t))=(x_1,x_2,x_3,x_4)\in V$. 
First, we need to determine $a,b,c$ of a curve $C_{a,b,c}$ given $S$. Second, we will obtain the explicit parametrization of $(z(t),w(t))\in C$. And finally, we will get  formulas for $\psi(\phi(t))\in V$.

We followed the same approach as Fouque-Tibouchi~\cite{ft2012} (FT) for converting $S$ into a conic $C$ due to two reasons. It is expected that our derivation leads to the same formulas as the ones of FT when they are instantiated with a BN curve; and, the FT paper  already includes a detailed analysis of the image size of the map and the proofs that this encoding is admissible, which is required to get indifferentiability.
\subsection{Defining $\phi$}
Let $S$ as above, and fix $u\in\mathbb{F}$ as a variable subject to some restrictions given in the course of this description.
Let's manipulate $S$ to transform it to a conic $C$.
\begin{eqnarray*}
\lambda^2(u^2+uv+v^2+A) &=& -f(u) \\
 \lambda^2\left( \frac{3}{4}u^2 + \left(v+\frac{u}{2}\right)^2 \right) &=& -f(u)-A\lambda^2
\end{eqnarray*}
Define $z=v+\frac{u}{2}$ and $w=\frac{1}{\lambda}$, and replace them into the previous equation.
\begin{eqnarray*}
 \lambda^2\left(\frac{3}{4}u^2 + z^2\right)  &=& -f(u)-A\lambda^2 \\
 \frac{3}{4}u^2 + z^2  &=& -\frac{f(u)}{\lambda^2}-A \\
  z^2 + f(u) w^ 2  &=& -\left(\tfrac{3}{4}u^2+A\right)
\end{eqnarray*}
Hence, we have the shape of a curve $C$ with coefficients $a=1$, $b=f(u)$, and $c=-\left(\tfrac{3}{4}u^2+A\right)$. We want that $C$ be non-degenerated ($a,b,c\neq 0$), so \textcolor{red}{$f(u)\neq 0$} and \textcolor{red}{$3u^2+4A \neq 0$}.

At this point, we have an explicit conic $C$. Now, we want to derive a parametrization of their points. To do that, we know there exists a point $(z_0=\sqrt{\frac{c}{a}},0)\in C$, iff $\frac{c}{a}$ is a QR. Thus, define $z_0=\sqrt{-\left(\tfrac{3}{4}u^2+A\right)} =\tfrac{1}{2}\sqrt{-(3u^2+4A)}$ and we must guarantee that there exists a $u$ such that \textcolor{red}{$-(3u^2+4A)$ is a QR}.

The parametrization of $C$ is given as: $(z(t),w(t)) = \left( z_0+tw(t),-\frac{2az_0t}{at^2+b} \right)$, where
\begin{eqnarray*}
 w(t) &=& -\frac{2z_0t}{t^2+f(u)}\\
 z(t) &=& z_0-\frac{2z_0t^2}{t^2+f(u)}
\end{eqnarray*}

Now, lets derive the explicit map $\phi$, which sends $ t \mapsto (u,v(t),\lambda(t))$ to a point in $S$. We know that $z=v+\tfrac{u}{2}$ and $w= \tfrac{1}{\lambda}$, lets solve these equations for $v(t)$ and $\lambda(t)$.
\begin{eqnarray*}
 v(t) &=&  z(t)-\frac{u}{2} = z_0-\frac{2z_0t^2}{t^2+f(u)}-\frac{u}{2}
       =  -\frac{u}{2}-z_0\left(\frac{t^2-f(u)}{t^2+f(u)} \right) \\
\lambda(t) &=& \frac{1}{w(t)} = -\frac{t^2+f(u)}{2z_0t} \\
\end{eqnarray*}
At this point, we require that both maps be defined, hence, we have:
\begin{itemize}
 \item $v(t)$ is defined when \textcolor{blue}{$t^2+f(u)\neq 0 $}.
 \item Since $t^2+f(u)\neq 0 $, then $\lambda(t)\neq 0$.
 \item $\lambda(t)$ is defined when $2z_0t\neq 0 $: We already know $z_0\neq 0$. So, we require \textcolor{blue}{$t\neq 0$}.
\end{itemize}


\subsection{Defining $\psi\circ\phi$}
Now, lets derive the explicit map $\psi\circ\phi$, which sends $t\mapsto(x_1(t),x_2(t),x_3(t),x_4(t))\in V$ to a point in the treefold.
\begin{equation}
 \begin{aligned}
 x_1(t) &= -\frac{u}{2}-z_0\left(\frac{t^2-f(u)}{t^2+f(u)} \right) \\
 x_2(t) &= -u-x_1  \\
 x_3(t) &=  u +\frac{1}{4{z_0}^2} \frac{\left(t^2+f(u)\right)^2}{t^2} 
 \end{aligned}
\end{equation}
These equations gives three candidates to be the $x$-coordinate of a point on $E$, where the unique restrictions in the parameter $t$ are \textcolor{blue}{$t^2+f(u)\neq 0 $} and \textcolor{blue}{$t\neq 0 $}.

\subsection{Solving the Map Exceptions}
There exist some values $t$ such that violates the \textcolor{blue}{blue} restrictions. However, we want the hashing works for any $t\in\mathbb{F}$. To remedy that, we rely on the function
\begin{equation}
 \begin{aligned}
\texttt{inv0}\colon \mathbb{F}&\rightarrow\mathbb{F}\\
x&\mapsto 1/x \\
0&\mapsto 0
 \end{aligned}
\end{equation}
Then, we can apply \texttt{inv0} on the calculation of $x_i$, and observe its behaviour under all combinations of blue restrictions:
\begin{center}
\begin{tabular}{l|l|l}
 $t^2+f(u)   = 0$ & $t\neq 0 $ & $x_1(t)=x_2(t)=-\frac{u}{2}\,,\;x_3(t)=u$ \\[7pt] \hline
 $t^2+f(u)\neq 0$ & $t   = 0 $ & $x_1(t)=-\frac{u}{2}+z_0\,,\;x_2(t)=-\frac{u}{2}-z_0\,,\;x_3(t)=u$ \\[7pt] \hline
 $t^2+f(u)   = 0$ & $t   = 0 $ & Can't happen, since $f(u)\neq0$. \\
\end{tabular}
\end{center}
Also, it is easy to show that $f(-\frac{u}{2}+z_0) = f(u)$.
Hence, by choosing $u$ such that \textcolor{red}{$f(u)$ is a QR}, it always returns a point in the curve for any $t\in\mathbb{F}$.

\section{SW Algorithm}

\subsection{Requirements}
Given an elliptic curve $E_{A,B}/\mathbb{F}$, find $u$ under the following restrictions:
\begin{itemize}
\begin{multicols}{2}
 \item $f(u)$ is a non-zero QR.
 \item $-(3u^2+4A)$  is a non-zero QR.
\end{multicols}
\end{itemize}

\subsection{Constants}
Once such a $u$ was found, precompute the following constants.
\begin{itemize}
\begin{multicols}{2}
 \item $c_0 = -u/2$
 \item $c_1 = f(u) = u^3+Au+B$
 \item $c_2 = -z_0 = -\frac{1}{2}\sqrt{-\left(3u^2+4A\right)}$
 \item $c_3 = \dfrac{1}{4{z_0}^2} $ 
\end{multicols}
\end{itemize}

\subsection{Implementation}
Assuming these are operations on $\mathbb{F}$:
\begin{itemize}
\begin{multicols}{3}
 \item \textbf{M} = multipication
 \item \textbf{S} = squaring
 \item \textbf{A} = addition/subtraction
 \item \textbf{I} = Inverse
 \item \textbf{E} = Exponentiation
 \item \textbf{L} = Legendre symbol
 \item \textbf{R} = square-root 
\end{multicols}
\end{itemize}
The implementation of SW map takes 1\textbf{I}+2\textbf{L}+1\textbf{R}+10\textbf{M}+5\textbf{S}+9\textbf{A} field operations, which is around 4\textbf{E} field exponentiations.

\begin{algorithm}[H]
\caption{SW Map}
  \begin{algorithmic}[1]
  \small
 \ENSURE $t\in \mathbb{F}$
 \REQUIRE  $(x,y)\in E_{A,B}/\mathbb{F}$
%  \begin{multicols}{1}
 \STATE $t_0 \leftarrow t^2$
 \STATE $t_1 \leftarrow t_0+c_1$
 \STATE $t_2 \leftarrow t_0-c_1$
 \STATE $t_3 \leftarrow t_0\times t_1$
 \STATE $t_4 \leftarrow \text{inv0}(t_3)$
 \STATE $x_1 \leftarrow c_0+c_2\times t_2\times t_4\times t_0$
 \STATE $x_2 \leftarrow -u-x_1 $
 \STATE $x_3 \leftarrow u+c_3\times {t_1}^2\times t_4\times t_1 $
 \STATE $f_1 \leftarrow x_1\times({x_1}^2+A)+B$
 \STATE $f_2 \leftarrow x_2\times({x_2}^2+A)+B$
 \STATE $f_3 \leftarrow x_3\times({x_3}^2+A)+B$
 \STATE $b_1\leftarrow 0$\,, $b_2\leftarrow 0$\,, $s\leftarrow 0$
 \IF{$f_1$ is QR}
    \STATE $b_1 \leftarrow 1$
 \ENDIF
 \IF{$f_2$ is QR} 
    \STATE $b_2 \leftarrow 1$
 \ENDIF
 \STATE $x\leftarrow \textsc{cmov}(x_3,x_2,b_2)$
 \STATE $x\leftarrow \textsc{cmov}(x,x_1,b_1)$
 \STATE $f\leftarrow \textsc{cmov}(f_3,f_2,b_2)$
 \STATE $f\leftarrow \textsc{cmov}(f,f_1,b_1)$
 \STATE $y \leftarrow \sqrt{f} $
 \IF{$\text{sgn0}(t) \neq \text{sgn0}(y) $}
    \STATE $s\leftarrow 1$
 \ENDIF
 \STATE $y\leftarrow \textsc{cmov}(-y,y,s)$
%  \end{multicols}
 \RETURN $(x,y)$
 \end{algorithmic}
\end{algorithm}

\section{Examples}

\subsection{BN Curves}
Setting $u=1$ for BN curves leads to Fouque-Tibouchi~\cite{ft2012} original formulas.

\bibliographystyle{ieeetr}
\bibliography{ecc.bib}

\appendix
\section{Parametrizing a Conic}
\label{app:conic}

We want to find a  parametrization of the points on $C$ on variable $t$.
To do that, we intersect a line passing through $(z_0,w_0)$, this line has equation:
\begin{equation}
 z = t (w-w_0)+z_0
\end{equation}
with the slope $t$.
It is clear that $(z_0=\sqrt{\frac{c}{a}},0)\in C$ is a point as long as $\frac{c}{a}$ be a QR. Then, the line equation passing through $(z_0,0)$ is $z=z_0+tw$.

Now, we substitute $z$ into $C$ equation as follows:
\begin{eqnarray*}
 az^2+bw^2 = c\\
 a(z_0+tw)^2+bw^2- c = 0 \\
 a{z_0}^2+2a{z_0}tw+t^2w^2+bw^2 - c = 0
\end{eqnarray*}
Solving this equation for $w$, we have:
\begin{eqnarray*}
 a{z_0}^2+2a{z_0}tw+at^2w^2+bw^2 - c = 0 \\
 w( at^2w+bw+2a{z_0}t)-c+a{z_0}^2 = 0 \\
 w( at^2w+bw+2a{z_0}t)= 0 \\
 w( w( at^2+b) + 2a{z_0}t)= 0 
\end{eqnarray*}
hence $w=0$ or $w = -\dfrac{2atz_0}{at^2+b}$.
Finally, we have that
\begin{equation}
(z(t),w(t)) = \left( z_0+tw(t),-\dfrac{2atz_0}{at^2+b} \right)\in C 
\end{equation}
for $z_0=\sqrt{\frac{c}{a}}$ be a QR.
\end{document}


